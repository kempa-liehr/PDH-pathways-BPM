\section{Discussion from Thesis}
The aim of this study is to use business process modelling methods to design a process mining pipeline that supports discovery, conformance analysis, and enrichment of healthcare pathways from hospital health records, and to apply the designed pipeline to two clinical case studies. Results from the two case studies indicate that the designed pipeline with ProM as the main process mining tool is effective for mining both simple and complex healthcare pathway models to produce a concise form that is easy for clinical interpretation. This chapter discusses the advantages and limitations of the proposed healthcare pathway mining method for the three stages of analysis (i.e. healthcare pathway discovery, conformance, and enrichment). Future research directions that would maximize this study’s potential for industrial implementation are also discussed.

The discovery stage of the designed process mining pipeline provides an unbiased view of real patient experiences through the treatment process. The proposed approach to pathway discovery is data-based, so the accuracy of the output healthcare pathway model is highly dependent on data accuracy. Incorrectly logged health records in hospital information systems can lead to an inaccurate representation of real patient experiences through the treatment process, so clinicians must be consulted to determine if unexpected results are caused by faulty data. The main advantage of the proposed approach is that healthcare pathway visualization is automatic, and this means that the healthcare pathway mining pipeline has the potential to be fully automated in the future. Health records from the same hospital usually follow systematic formats and notations, so the process of converting electronic health records to clinical event logs compatible with business process mining software can be automated. The main challenge to automating the entire healthcare pathway discovery process is the event log processing steps required to reduce clinical variations. Even though the chapter on healthcare pathway mining methodology (Chapter 2) provides guidelines on processing steps that are effective for reducing unnecessary clinical variations in event logs, a certain level of clinical input is still required to ensure that no crucial information is filtered. In the first case study (see Chapter 3), appendicitis and cholecystitis pathways both contain repetitive activity patterns (i.e. antibiotics and monitoring labs) that result in incomprehensible spaghetti-models, and these patterns are identified by manual inspection of the pathway variants. There is potential for the process mining pipeline proposed in this study to be combined with clinical pattern mining algorithms developed in previous studies (e.g. the SCP-Miner and CCP-Miner developed by Huang, Lu, and Duan [16]) so that repetitive activity patterns can be identified and condensed automatically. Further research is required to determine the compatibility of the proposed process mining pipeline with various pattern mining algorithms.

The appendicitis and cholecystitis pathway models, from the first clinical case study, and the ambulatory cardiac care pathway model, from the second clinical case study, are all confirmed by clinicians to be easy to interpret and clinically meaningful, but one major limitation of the proposed pathway discovery approach is that the discovered healthcare pathway models can contain pathways that are not executed by any patient traces in the clinical event log. This is because the process mining software arranges all activities into a coherent pathway model by a configuration that summarizes all pathway variants, but a configuration that summarizes pathway variants does not guarantee that all pathways in the final healthcare pathway model make sense clinically. This also means that the pathway representation of models discovered by the proposed pipeline is unable to accurately capture all relations between activities. For example, the ambulatory cardiac care pathway model from the second case study (see Figure 56) shows that it is possible for patients to perform activity ‘Coronary Angiogram’ without performing ‘First Cardiology’ first. Close inspection of the event log revealed that patients who perform ‘Coronary Angiogram’ always perform ‘First Cardiology’ first, but this relation between the two activities is not evident in the discovered pathway model. Analysis of pathway variants instead of the pathway model does portray the true relations between all activities but this is not a realistic approach for complex healthcare pathways. For relatively simple healthcare pathways like the appendicitis and cholecystitis case study, patient traces can be organized into a manageable number of unique pathway variants and all variants can be analyzed individually. For complex healthcare pathways like the ambulatory cardiac care case study, patient traces cannot be organized into a manageable number of pathway variants because the level of variation between patients is too high. The healthcare pathway discovery stage of the methods presented here provides a concise overview of the fundamental structure of the overall pathway but does not always effectively capture the true relations between clinical activities. 
The conformance analysis stage of the designed process mining pipeline effectively identifies and visualizes patient deviations from the discovered healthcare pathway models. Even though this study only analyzes patient conformance to the discovered healthcare pathways, the same method could be applied analyze patient conformance to predefined ideal pathways or new pathway designs. The proposed approach to conformance analysis could be extended in the future to a monitoring tool to maintain quality of care for all patients. It is very difficult for clinicians to manually track individual patient traces through the treatment process and ensure that they are conforming to standard protocols. Identifying unwarranted deviations and making the required interventions early in the process has the potential to improve health outcomes and decrease cost. Even though the proposed approach effectively visualizes patient deviations, it cannot distinguish between clinically significant deviations and minor deviations that have no influence on patient health outcome. For example, some deviations are the results of scheduling availability and the order in which a required set of clinical activities are executed is not always relevant to health outcome. The proposed conformance analysis approach cannot automatically identify clinically significant deviations, and reasons for all deviations must be investigated manually. This is a major obstacle to future hospital implementation of the proposed pipeline, because it is inefficient and impractical to alert clinicians to minor patient deviations. Further research is required to develop a more specialized tool for conformance analysis that is able to classify patient deviations according to their association with patient health outcome.

The enrichment stage of the designed process mining pipeline is effective for identification of bottlenecks and treatment delays in healthcare pathways. The enrichment stage of the ambulatory cardiac care case study revealed that waiting times for transthoracic echo are exceedingly long (see §4.7), and resource reallocation is recommended to reduce the treatment delay. The effects of the delay on patient health outcome can be further investigated by computer simulations, but more clinical data is required for this approach. Analysis of waiting times and hospital length of stays is based on the assumption that the occurring timestamps of clinical activities are accurate. North Shore Hospital has an automated information system that stores patient records in real time and accuracy of the timestamps in the two case studies is therefore high, but the precision of activity timestamps is not always consistent across hospital systems. For example, one clinical department could record the exact time of day an activity is executed, while another department only records the date of execution. This leads to a margin of error of 24 hours, and this margin of error must be taken into account when interpreting the results. Accuracy and precision of clinical data must be assessed when the proposed pipeline is applied to new clinical case studies.

Analysis of pathway performance based on pathway variants reveals associations between clinical activities and performance indicators, but associations do not necessarily indicate causation. For example, the appendicitis pathway variants from the first case study that consist of postoperative imaging activities generally have longer postoperative length of stays, even though larger samples are required to confirm this connection (see §3.9). One possible explanation is that appendicitis patients that develop extra complications are more likely to require postoperative imaging activities. These are also the patients that remain under monitoring for a longer time period and hence have a delayed discharge date. Similarly, the ambulatory cardiac care pathway variants from the second case study that consist of visits to the emergency department generally have higher mortality rates (see §4.84.8). In both of these cases, patient condition is the main factor that influences the pathway performance indicators.
Dr. Andreas W. Kempa-Liehr’s work on developing probabilistic models to predict postoperative length of stay using information from the discovered healthcare pathway has the potential to improve the efficiency of hospital scheduling and resource allocation. This also shows that the proposed process mining pipeline has the potential to support the development of machine learning models for the prediction of performance indicators. A future improvement that could be made to the proposed pipeline is the incorporation of clinical diagnosis into the healthcare pathway models. Incorporating clinical diagnosis into pathway models allows for the identification of critical decision points where pathway variants diverge, but the main challenge is obtaining access to clinical diagnostic data. Overall, there is potential for the use of business process modelling techniques to improve the method of analysis of healthcare pathways, but the capability of other more specialized business process mining software should also be explored.