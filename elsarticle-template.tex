% Targeted for International Journal of Medical Informatics
% Word limit: 3000
%\documentclass[review]{elsarticle}
\documentclass{elsarticle}


\usepackage{lineno,hyperref,graphicx}
\usepackage[table]{xcolor}
\usepackage[section]{placeins}
\modulolinenumbers[5]
\graphicspath{ {./images/} }

\journal{Elsevier}

%%%%%%%%%%%%%%%%%%%%%%%
%% Elsevier bibliography styles
%%%%%%%%%%%%%%%%%%%%%%%
%% To change the style, put a % in front of the second line of the current style and
%% remove the % from the second line of the style you would like to use.
%%%%%%%%%%%%%%%%%%%%%%%

%% Numbered
%\bibliographystyle{model1-num-names}

%% Numbered without titles
%\bibliographystyle{model1a-num-names}

%% Harvard
%\bibliographystyle{model2-names.bst}\biboptions{authoryear}

%% Vancouver numbered
%\usepackage{numcompress}\bibliographystyle{model3-num-names}

%% Vancouver name/year
%\usepackage{numcompress}\bibliographystyle{model4-names}\biboptions{authoryear}

%% APA style
%\bibliographystyle{model5-names}\biboptions{authoryear}

%% AMA style
%\usepackage{numcompress}\bibliographystyle{model6-num-names}

%% `Elsevier LaTeX' style
\bibliographystyle{elsarticle-num}
%%%%%%%%%%%%%%%%%%%%%%%

\begin{document}

\begin{frontmatter}

\title{Healthcare Pathway Discovery, Conformance, and Enrichment}

%% Group authors per affiliation:
%\author{Michael O’Sullivan,  Andreas W. Kempa-Liehr, Christina Lin\fnref{myfootnote}, Randall Britten, Delwyn Armstrong}
%\address{Radarweg 29, Amsterdam}
%\fntext[myfootnote]{Since 1880.}

%% or include affiliations in footnotes:
\author[mymainaddress]{Christina Lin}
\author[mymainaddress]{Andreas W. Kempa-Liehr\corref{mycorrespondingauthor}}
\cortext[mycorrespondingauthor]{Corresponding author}
\ead{a.kempa-liehr@auckland.ac.nz}

\author[Orion]{Randall Britten}
\author[Waitemata]{Delwyn Armstrong}
\author[Waitemata]{Jonathan Wallace}
\author[Waitemata]{Patrick Gladding}
\author[mymainaddress]{ Michael O'Sullivan}

\address[mymainaddress]{Department of Engineering Science, The University of Auckland, 70 Symonds St, Grafton, Auckland, New Zealand}
\address[Orion]{Orion Health, 181 Grafton Rd, Grafton, Auckland, New Zealand}
\address[Waitemata]{Waitemata District Health Board, 124 Shakespeare Rd, Takapuna, Auckland, New Zealand
}

\begin{abstract}
\subsection*{Background and purpose}
Healthcare pathways define the execution sequence of clinical activities as patients move through a treatment process, and they are critical for maintaining quality of care. The aim of this study is to investigate the utilization of business process modelling (BPM) to design an adaptive healthcare pathway mining methodology, with particular emphasis on producing pathway models that are easy to interpret for clinicians without a sufficient background in process mining.

\subsection*{Method}
This study utilizes business process-mining software (ProM) to design a process mining pipeline for healthcare pathway discovery, conformance analysis, and enrichment using hospital records. The efficacy of the BPM approach is demonstrated via two case studies that apply the proposed process mining pipeline to discover appendicitis and cholecystitis pathways from hospital records. Machine learning methods are utilised to explore pathway features that influence patient recovery time.

\subsection*{Results}
The produced appendicitis and cholecystitis pathway models are easy for clinical interpretation and provide an unbiased overview of patient movements through the treatment process. Analysis of the discovered pathway models enables reasons for longer than usual treatment times to be explored and deviations from standard treatment pathways to be identified. A probabilistic regression model that estimates patient recovery time based on the information extracted by the process mining pipeline is developed and has the potential to be very useful for hospital scheduling purposes.

\subsection*{Conclusion}
This study establishes the application of the business process modelling tool, ProM, for the improvement of healthcare pathway mining methods. There is also value in investigating the capabilities of other business process modelling tools for healthcare pathway mining purposes. The proposed mining pipeline also has the potential to support the development of machine learning models to further relate healthcare pathways to performance indicators such as readmission rates and mortality rates. 

\end{abstract}

\begin{keyword}
Healthcare pathway; Process mining; Electronic health record
%\MSC[2010] 00-01\sep  99-00
\end{keyword}

\end{frontmatter}

\linenumbers

\section{Introduction}
Healthcare pathways are critical for reducing clinical variability, affecting operational excellence, and thereby maximizing health outcomes \cite{Lin2001}. They define the execution sequence of clinical activities as patients move through a treatment process, a department, a hospital, or a wider health organization (e.g., a District Health Board) \cite{Huang2016}. The accurate definition of healthcare pathways and patient conformance to those pathways is an issue of increasing relevance as precision medicine enables targeted approaches and diagnostic splitting. The proliferation of pathway branches is exponential, and pathways are increasingly non-linear. 

Most healthcare pathways result from clinician-led practice rather than explicit pathway design via a consensus model and systems approach. In addition, healthcare pathways “shift” dynamically as steps in the pathway are altered or resources change along the pathway. If no explicit redesign of pathways is performed, then the providers of the pathways (and its associated resources) may be unaware of the change \cite{Zhang2015}. Pathway discovery (identification of pathways without a priori knowledge), conformance analysis (including gaps in care and clinical variability) and pathway enrichment (enrichment of a priori models with additional event data) are critical for healthcare services now, and into the future \cite{Baker2017}.

Past studies have shown that there is potential for informative healthcare pathways to be extracted from hospital health records \cite{Xu2017}, \cite{Iwata2013}, but there is currently no consensus on a systematic healthcare pathway mining method that supports explicit design and conformance analysis of concise and comprehensible healthcare pathway models. The research described in this paper investigates the utilization of Business Process Modelling (BPM), outlined by Becker, Rosemann, and Von Uthmann \cite{Becker2000}, to provide a scaffold for healthcare pathway discovery, conformance analysis and enrichment. The main objectives of applying BPM to healthcare data include:
\begin{enumerate}
    \item Pathway discovery
    \begin{itemize}
        \item Investigate the potential of ProM (a process-mining software package) to discover healthcare pathways from hospital records \cite{VanDongen2005}.
    \end{itemize}
    \item Conformance analysis
    \begin{itemize}
        \item Apply BPM conformance analysis to discovered healthcare pathways. 
        \item Improve detection of possible non-conformance and explain anomalies.
    \end{itemize}
    \item Data enrichment
    \begin{itemize}
        \item Investigate correlation between healthcare pathway and performance-determining factors (e.g., patient length-of-stay, readmission rate).
    \end{itemize}
\end{enumerate}

\section{Healthcare Pathway Mining Methodology}
This chapter outlines the proposed process mining pipeline designed for mining healthcare pathways from hospital records using business process modelling tools. This study adopts the scientific computing practices recommended by Wilson et al. \cite{Wilson2014}, \cite{Wilson2017} to ensure that all results are reproducible. The proposed process mining pipeline consists of three major sections that correspond to each of the three main objectives (i.e. pathway discovery, conformance, and enrichment). An overview of the process mining pipeline designed for this study is shown in Fig.~\ref{fig:pipeline}. This chapter discusses each step of the designed pipeline in detail.

\begin{figure}[t]
\centering
\includegraphics[width=\textwidth]{images/pipeline_diagram_journal.jpg}
\caption{Overview of the process mining pipeline designed based on business process modelling methods.}
\label{fig:pipeline}
\end{figure}

This study chooses ProM (version 6.7) as the main process mining tool. ProM is an open-source process mining software that is effective for construction of business models from input data files. The use of ProM in the healthcare sector has not been thoroughly explored in the past, but Van Der Aalst et al. \cite{VanDerAalst2007} has demonstrated the applicability of the software for industrial processes. ProM is chosen for this study because it has an intuitive user interface, and supports many process mining plug-ins \cite{VanDongen2005}. The process mining plug-ins supported by ProM are well documented. All these features of ProM make it easy for the process mining steps in the pipeline (see Fig.~\ref{fig:pipeline}) to be repeated by users with no background in process mining.

\subsection{Healthcare Pathway Discovery}
Healthcare pathway discovery is the first stage of the proposed process mining pipeline. The aim is to use patient healthcare records stored in hospitals’ information systems to design a concise pathway model that is easy for clinical interpretation.

\subsubsection{Clinical Data Processing}
Healthcare pathways generally have much higher levels of complexity than standard business processes, and unprocessed clinical data contains too many clinical variations for a clean and concise pathway model to be mined \cite{Huang2013}, \cite{Veiga2010}. A pathway variant is a unique activity sequence of a complete patient trace. The ProM plug-in \texttt{`Explore Event Log'} organizes patient traces into pathway variants, and the total number of pathway variants is also an indicator of the level of clinical variation between patient traces. 
There are several ways to reduce clinical variations. Pathway variants visualized by the plug-in \texttt{`Explore Event Log'} are examined closely to determine the most suitable processing methods. Clinical input is highly recommended at this step particularly for complex or unfamiliar healthcare pathways. There are three effective methods for reducing clinical variations without filtering patient traces:

\begin{enumerate}
    \item Cluster clinical activities\\
        Cluster clinical activities that are similar in nature so that the range of activities is reduced to a manageable size, e.g., `Abdomen CT scan' and `Pelvis CT scan' could be clustered into a single activity under `CT scan'.
    \item Merge consecutive clinical activities\\
        Merge series of identical activities that are performed consecutively into a single activity, e.g., a patient receiving the same medication five times on the same day could be regarded as a single activity.
    \item Condense repetitive activity patterns\\
        Activity sets that repeat, but with variable cycle length, indicate an activity that must be performed periodically while the patient is waiting for a different activity to begin, e.g., lab tests to monitor a patient’s condition, medication to prevent infection. These repetitive patterns could be condensed into a single, parallel activity.
\end{enumerate}

Clinical data processing and pathway visualization are conducted iteratively until a concise model is produced. Appropriate processing methods must be chosen in context of the clinical case and clinical input is critical to the selection of appropriate processing methods. 

\subsubsection{Pathway Visualization}
ProM offers a plug-in known as \texttt{`Inductive Visual Miner'} that visualizes healthcare pathway models from input clinical data almost instantaneously. It is the primary process-mining tool used in this study to construct healthcare pathway models. Pathway models constructed by \texttt{`Inductive Visual Miner'} are in the form of process trees. A process tree is a hierarchical map comprised of decision nodes and `tasks' representing clinical activities \cite{25a7fd818bf44606a903d9b78b95cdd3}. Healthcare pathway models constructed by this plug-in allow for the identification of healthcare pathway branches throughout the process.

\subsection{Healthcare Pathway Conformance Analysis}
Conformance analysis evaluates the degree to which a healthcare pathway model captures movement of patient traces by comparing the discovered model to real clinical data. 
Accuracy of the discovered healthcare pathway model is validated if the majority of the patient traces conform to the model. Patient traces rarely all follow identical pathways, so the healthcare pathway model is not expected to capture all patient traces. The objective is to discover a healthcare pathway model that captures the fundamental structure of most patient traces. 

ProM offers tools for conformance analysis of a discovered healthcare pathway model. Plug-in \texttt{`Inductive Visual Miner'} compares patient traces from input clinical data to the healthcare pathway model and visualizes patient deviations from the pathway model. If valid patient traces deviate from a healthcare pathway model, then adjustments are made to the model to improve patient conformance, e.g., if a new form of treatment has been introduced, but not yet included in the model. Conversely, if invalid patient traces don't conform to a healthcare pathway model, then the conformance analysis can identify where these invalid patient traces deviate from the model and investigate the reason for the discrepancy, e.g., clinicians following obsolete pathways, data errors, etc.

\subsection{Healthcare Pathway Data Enrichment}
\subsubsection{Evaluate Healthcare Pathway Performance}
The main objectives of evaluating healthcare pathway performance are to understand the strengths and weaknesses of current pathway design, and to identify potential methods of improvement. Possible indicators of healthcare pathway performance include waiting times of clinical activities, hospital length of stay, recovery time and readmission rates. Most of these indicators can be calculated or estimated using standard clinical timestamps.

For surgical healthcare pathways, hospital length of stay is one of the critical indicators for evaluating healthcare pathway performance. Hospital length of stays are analyzed based on pathway variants because patients that follow identical activity sequences are likely to have similar clinical diagnoses. One approach is to analyze length of stay at each step of the pathway variants (see section 3.3 for application example). This allows for potential rate-determining steps to be identified because the contributions of all clinical activities to total length of stay is clear.

\subsubsection{Regression Model for Healthcare Pathway Performance}
Machine learning techniques are used to explore factors that influence healthcare pathway performance indicators. All information derived from the discovered healthcare pathways such as time-stamped clinical activities and patient demographics are all considered as explanatory variables to build regression models for performance indicators. Performance indicators such as hospital length of stay and patient readmissions are all dependent on human decisions, so regression models are not expected to have powerful predictive abilities.  The aim is to investigate if certain clinical decisions are likely to improve pathway performance, either by shortening hospital length of stay or decreasing probability of readmission.

\section{Case Studies: Appendicitis and Cholecystitis}
The two case studies apply the designed process mining pipeline (see section 2) to discover appendicitis and cholecystitis pathways from hospital records. Two years’ worth of data, from 2015 to 2017, on 448 appendicitis patients and 52 cholecystitis patients are collected from North Shore Hospital for this study. 

\subsection{Appendicitis Pathway Variants}
This section shows the appendicitis pathway variant plot auto-generated by ProM. Appendicitis pathway variants are analyzed without activities representing antibiotics (i.e. pre-operation and post-operation cefuroxime/metronidazole) because the clinicians confirmed that antibiotics are usually taken while the patient is waiting for surgery or discharge. The duration of these activities are therefore highly variable and result in a high number of unique pathway variants.

The appendicitis pathway model consists of 13 pathway variants. The 13 pathway variants from the appendicitis pathway model are shown in Fig.~\ref{fig:appendicitis pathway variants}, and the number of patient traces that follow each pathway variant are listed in Table \ref{table:appendicitis variant table}. This plot is generated automatically by ProM tools. All clinical activities are represented by a start event and a stop event, and they are colour coded such that the same colour refers to the same clinical activity. Pathway variants from Fig.~\ref{fig:appendicitis pathway variants} are ordered from the most frequent (index 0) to the least frequent (index 12). The most frequent pathway variant (index 0) only consists of anesthesia and surgery, while the second most frequent variant (index 1) also includes pre-operation X-ray.

\begin{figure}[t]
\hspace{-2cm}
\includegraphics[width=1.5\textwidth]{images/appendicitis_variant_index_anes.jpg}
\caption{Appendicitis pathway variants auto-generated by ProM and appended with a legend. The top three pathway variants account for approximately 70\% of the patient traces. The statistics on the left are not readable in this reproduction but are listed in Table \ref{table:appendicitis variant table}.}
\label{fig:appendicitis pathway variants}
\end{figure}
\clearpage

\begin{table}[t]
\centering
\caption{Number of patient traces that follow each appendicitis pathway variant.}
\label{table:appendicitis variant table}
\begin{tabular}{ l l l }
 \hline
 Index & Number of Patient Traces & Percentage of Patients \% \\ 
 \hline
 0 & 148 & 33.04\\ 
 \hline
 1 & 86 & 19.20\\ 
 \hline
 2 & 82 & 18.30\\ 
 \hline
 3 & 78 & 17.41\\ 
 \hline
 4 & 29 & 6.47\\ 
 \hline
 5 & 6 & 1.34\\ 
 \hline
 6 & 6 & 1.34\\ 
 \hline
 7 & 3 & 0.67\\ 
 \hline
 8 & 2 & 0.45\\ 
 \hline
 9 & 2 & 0.45\\ 
 \hline
 10 & 2 & 0.45\\ 
 \hline
 11 & 2 & 0.45\\ 
 \hline
 12 & 2 & 0.45\\ 
 \hline
\end{tabular}
\end{table}

\subsection{Appendicitis Pathway Model}
The first stage of the appendicitis pathway model visualized by \texttt{`Inductive Visual Miner'} from processed clinical data is shown in Fig.~\ref{fig:ivm pathway model example}. Unlike the pathway variants, the appendicitis pathway model incorporates activities representing antibiotics. The model indicates that 42 patients perform ultrasound and 183 patients perform X-ray upon admission. Please refer to Leeman's manual on \texttt{`Inductive Visual Miner'} for details on the model notations used in  Fig.~\ref{fig:ivm pathway model example} \cite{leemansinductive}.

\begin{figure}[t]
\includegraphics[width=\textwidth]{images/ivm_appendicitis_first_stage_example.png}
\caption{First stage of the appendicitis pathway model generated by ProM. There are 214 patients that enter the first stage of the treatment pathway, which consists of X-ray, or ultrasound, or both.}
\label{fig:ivm pathway model example}

\includegraphics[width=\textwidth]{images/communicative_appendicitis_process_models_anes.jpg}
\caption{Appendicitis pathway model. All pre-operation and post-operation activities belong to radiology and pharmacy departments. Pre-operation antibiotics are taken in the second stage of the treatment pathway.}
\label{fig:appendicitis pathway model}
\end{figure}

The ProM notations for pathway models are complex and optimal for detailed analysis, so the appendicitis pathway model discovered by ProM is reformulated under new notations for easy clinical interpretation. The new model notations are summarized in Table \ref{table:notation table}, and the reformulated appendicitis pathway model is shown in Fig.~\ref{fig:appendicitis pathway model}. The section of the model labelled as `Stage 1' in Fig.~\ref{fig:appendicitis pathway model} corresponds to the same section shown in Fig.~\ref{fig:ivm pathway model example}. This is the final pathway model that has been compiled based on clinical input to account for valid patient deviations, and all patient traces conform to the updated pathway. The most frequent pathway variant (index 0) shown in Fig.~\ref{fig:appendicitis pathway variants} corresponds to the horizontal path from start to finish in Fig.~\ref{fig:appendicitis pathway model}.

\begin{table}[t]
\centering
\caption{Definitions of new pathway model notations.}
\label{table:notation table}
\begin{tabular}{ l c l }
 \hline
 Notation & Symbol & Definition \\ 
 \hline
 Orange Diamond 
 &
%\raisebox{-\totalheight}{\includegraphics[width=0.3\textwidth, height=60mm]{images/myLboro.png}}
 \raisebox{-3pt}{\includegraphics[width=0.5cm]{images/decision_node.png}}
 & Decision Point, indicating exclusive choice \\ 
 \hline
 Orange Connector 
 & 
 \raisebox{-3pt}{\includegraphics[width=0.5cm]{images/connection_node.png}}
 & Pathway Connection Point \\
 \hline
 Green Connector 
 & 
 \raisebox{-3pt}{\includegraphics[width=0.5cm]{images/start_node.png}}
 & Starting Point \\
 \hline
 Red Connector 
 & 
 \raisebox{-3pt}{\includegraphics[width=0.5cm]{images/finish_node.png}} 
 & Finishing Point \\
 \hline
\end{tabular}
\end{table}

\subsection{Appendicitis Patient Length of Stay}
This section and demonstrates the potential of pathway performance analysis based on the identified pathway variants. Patient length of stay at each stage of the appendicitis pathway measured from admission for all patient traces that follow the first pathway variant (index 0) and the third pathway variant (index 2) are shown in Fig.~\ref{fig:appendicitis length of stay variant 0} and Fig.~\ref{fig:appendicitis length of stay variant 2} respectively. These plots could be useful for identifying potential rate-determining steps in a healthcare pathway. One patient trace from Fig.~\ref{fig:appendicitis length of stay variant 0} and six patient traces from Fig.~\ref{fig:appendicitis length of stay variant 0} have unusually long surgery waiting times. The number of patient traces that follow each pathway variant is limited for this study, and more samples are required to confirm causes of delay along the appendicitis pathway. 

\begin{figure}[t]
    \centering
    \begin{minipage}{0.48\textwidth}
        \centering
        \includegraphics[width=\textwidth]{images/appendicitis_variant_length_of_stay_0_journal.jpg}
        \caption{Hospital length of stay of appendicitis patient traces following pathway variant 0. There is one potential outlier for surgery waiting time.}
        \label{fig:appendicitis length of stay variant 0}
    \end{minipage}\hfill
    \begin{minipage}{0.48\textwidth}
        \centering
        \includegraphics[width=\textwidth]{images/appendicitis_variant_length_of_stay_2_journal.jpg}
        \caption{Hospital length of stay of appendicitis patient traces following pathway variant 2. There are six potential outliers for surgery waiting time.}
        \label{fig:appendicitis length of stay variant 2}
    \end{minipage}
\end{figure}

\subsection{Generalized Regression Model for Post-operation Length of Stay}
Post-operation length of stay, measured from exiting theatre to discharge, can also be considered as recovery time and reflects the quality of pre-operation pathway design. Better understanding of post-operation length of stay using all the information available at the end of surgery could be useful for hospital scheduling purposes. Regression models are used to determine if certain clinical decisions are likely to shorten post-operation length of stay.

Pre-operation clinical activities, surgery timestamps, and patient demographics are all considered as explanatory variables, and post-operation length of stay is the target variable. The Python package `tsfresh' \cite{Christ2018} offers a function that selects statistically significant variables based on univariate hypothesis tests. A dummy regression model that always returns the mean of the target variable is built to provide a baseline for comparison. Several regression models are built and compared to the baseline using root mean squared error on cross-validation. The dummy regression model produces a root mean squared error of 29.97 (hours) on leave-one-out cross-validation.

A partial least squares regression model with 3 components is fit from 12 features, which have p-values smaller than 5\% without considering a correction of the false discovery rate. This model has a root mean squared error of 27.91 hours on leave-one-out cross-validation. The root mean squared error is only reduced by approximately two hours compared to the dummy regression model, so the partial least squares regression model does not have strong predictive ability. Statistically significant variables (considering a correction of the false discovery rate) that contributed most to the partial least squares regression model are listed as follows:

\begin{enumerate}
    \item Surgery start time
    \item Length of surgery
    \item Patient age
\end{enumerate}

The distributions of predicted post-operation length of stay and actual post-operation length of stay are compared in Fig.~\ref{fig:regression distribution}.

\begin{figure}[t]
\centering
\includegraphics[width=\textwidth]{images/akl_pls_regression_distribution.jpg}
\caption{Patient distributions of actual post-operation length of stay versus prediction.}
\label{fig:regression distribution}
\end{figure}

The partial least squares regression model does not significantly improve predictivity, and this suggests that the link between pre-operation activities and post-operation length of stay is minimal. No clinical decision that is likely to shorten post-operation length of stay is identified from generalized regression models.

\subsection{Probabilistic Model for Post-operation Length of Stay}
A probabilistic machine learning model which estimates the conditional probability distribution of post-operation length of stay for individual patients based on the available information at the end of surgery is developed. The model considers 26 features, from which 23 features are selected based on univariate hypothesis tests. Selection criteria for features is a p-value less than or equal to 10\% (for the univariate hypothesis tests). For the purpose of this machine learning model, no controlling procedure is used. Instead a partial least squares model with 2 components is fitted for the purpose of dimensionality reduction at each node of the probabilistic model.

This probabilistic model is inspired by the seminal work of M. Feindt \cite{Feindt2004}. The model decomposes the regression problem into a hierarchical set of classifications, which are designed as machine learning pipeline combining partial least squares discriminant analysis with logistic regression such that the conditional probability distribution can be estimated from a Bayesian decision tree. This model estimates the probability distribution of post-operation length of stay for individual patients in terms of 16 intervals of adaptive length reflecting the multi-modality of the appendicitis data set. Each interval summarizes the observations from approximately 25 samples. This algorithm is still in development, but the results are promising. The predicted probability distributions of four example patients are shown in Fig.~\ref{fig:probabilistic distribution}.

\begin{figure}[t]
\centering
\includegraphics[width=\textwidth]{images/akl_pls_probabilistic_distribution.png}
\caption{Predicted probability distributions of post-operation length of stay for four patients.}
\label{fig:probabilistic distribution}
\end{figure}

\subsection{Cholecystitis Pathway Variants}
This section shows the cholecystitis pathway variant plot auto-generated by ProM. Cholecystitis pathway variants are analyzed without activities from the `antibiotics' sub-process and the `monitoring labs' sub-process (see section 3.7 for the activities from the two sub-processes). Clinicians confirmed that these sub-processes are standard monitoring and maintenance systems while the patient is waiting for further diagnosis. Only analyzing activities from the primary cholecystitis pathway significantly reduces the level of clinical variation between patient traces.

The cholecystitis pathway model consists of 10 pathway variants. The 10 pathway variants from the cholecystitis pathway model are shown in Fig.~\ref{fig:cholecystitis pathway variants}, and the number of patient traces that follow each pathway variant are listed in Table \ref{table:cholecystitis variant table}. Pathway variants from Fig.~\ref{fig:cholecystitis pathway variants} are ordered from the most frequent (index 0) to the least frequent (index 9). The most frequent pathway variant (index 0) consists of anesthesia, surgery, and surgical pathology lab. The second pathway variant (index 1) includes surgery without anesthesia because of faulty clinical data.

\begin{figure}[t]
\hspace{-2cm}
\includegraphics[width=1.5\textwidth]{images/cholecystitis_variant_index_anes.jpg}
\caption{Cholecystitis pathway variants auto-generated by ProM and appended with a legend. The top three pathway variants account for approximately 63\% of the patient traces. The statistics on the left are not readable in this reproduction but are listed in Table \ref{table:cholecystitis variant table}.}
\label{fig:cholecystitis pathway variants}
\end{figure}

\begin{table}[t]
\centering
\caption{Number of patient traces that follow each cholecystitis pathway variant.}
\label{table:cholecystitis variant table}
\begin{tabular}{ l l l }
 \hline
 Index & Number of Patient Traces & Percentage of Patients \% \\ 
 \hline
 0 & 19 & 36.54\\ 
 \hline
 1 & 8 & 15.38\\ 
 \hline
 2 & 6 & 11.54\\ 
 \hline
 3 & 4 & 7.69\\ 
 \hline
 4 & 4 & 7.69\\ 
 \hline
 5 & 3 & 5.77\\ 
 \hline
 6 & 2 & 3.85\\ 
 \hline
 7 & 2 & 3.85\\ 
 \hline
 8 & 2 & 3.85\\ 
 \hline
 9 & 2 & 3.85\\ 
 \hline
\end{tabular}
\end{table}

\subsection{Cholecystitis Pathway Model}
The cholecystitis pathway model visualized by \texttt{`Inductive Visual Miner'} incorporates activities from the `antibiotics' sub-process and the `monitoring labs' sub-process. A breakdown of the reformulated cholecystitis pathway model into one primary pathway and two concurrent sub-processes is shown in Fig.~\ref{fig:cholecystitis pathway model}, and the model notations are summarized in Table \ref{table:notation table}. The first pathway model in Fig.~\ref{fig:cholecystitis pathway model} is the primary pathway, followed by the `antibiotics’ sub-process and the `monitoring labs’ sub-process. Patient traces can execute any combination of the two sub-processes concurrently with the primary pathway. The eight patient traces that follow the second pathway variant (index 1) do not conform to this pathway model because of faulty clinical data. Based on this model, pre-operation haematology and chemistry labs tend to span the entire pre-operation process, while pre-operation antibiotics are taken closer to surgery.

\begin{figure}[t]
\centering
\includegraphics[width=18cm,angle=270]{images/communicative_cholecystitis_process_models_anes.jpg}
\caption{Cholecystitis pathway model. The model is broken down into one primary pathway and two sub-processes.}
\label{fig:cholecystitis pathway model}
\end{figure}

\section{Conclusion}
Healthcare pathways are critical for maintaining quality of care and improving health outcome for all patients, but there is no consensus on a healthcare pathway mining pipeline suitable for hospital implementation that supports pathway discovery from hospital health records. Business process modelling methods are used to design a process mining pipeline that produces concise and comprehensible healthcare pathway models from hospital records, and supports conformance analysis and enrichment of the discovered pathways. The proposed process mining pipeline successfully constructs concise pathway models for the appendicitis and cholecystitis case studies. The produced healthcare pathway models are easy for clinical interpretation and provide an unbiased overview of real patient traces through the treatment process. Preliminary analysis on building a machine learning model to predict post-operation length of stay, using information extracted by the process mining pipeline, is showing promising results. This means that the proposed mining pipeline has the potential to support the development of machine learning models to further relate healthcare pathways to performance indicators. This study has established the use of business process modelling methods for the improvement of healthcare pathway mining methods, and there is value in investigating the capabilities of other business process modelling tools for healthcare pathway mining purposes.

\begin{table}[h]
\centering
\begin{tabular}{p{11cm}} 
 Summary points\\ 
 What was already known:
 \begin{itemize}
     \item Healthcare pathways are critical for reducing clinical variability, affecting operational excellence, and thereby maximizing health outcomes.
     \item  Most healthcare pathways result from clinician-led practice rather than explicit pathway design via a consensus model and systems approach. 
     \item  There is currently no consensus on a systematic healthcare pathway mining method that supports explicit design and conformance analysis of concise and comprehensible healthcare pathway models.
 \end{itemize}
 What this study adds:
 \begin{itemize}
     \item  The use of business process modelling methods improves the automatic mapping of healthcare pathways from clinical data.
     \item The application of business process modelling methods to healthcare pathway enables deviations from typical treatment pathways to be identified; and predictions for future treatment duration to be made, all using standard clinical data timestamps. 
 \end{itemize}
\end{tabular}
\end{table}

\section*{Author Contributions}
All authors have made substantial contribution to this study and approved the final manuscript.

\section*{Acknowledgements}
This research has been funded by the Precision Driven Health research partnership (project number 1209).

\section*{Declaration of Competing Interests}
The authors declare that they have no competing interests for this study.

\bibliography{mybibfile}

\end{document}